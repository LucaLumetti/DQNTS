\documentclass{article}
\input{structure.tex}
\usepackage[ruled,vlined]{algorithm2e}
\usepackage{hyperref}

\hypersetup{
    colorlinks=true,
    linkcolor=blue,
    filecolor=magenta,
    urlcolor=blue,
  }

\title{Automated Decision Making: Final Project}
\author{Luca Lumetti\\ \texttt{244577@studenti.unimore.it}}

\begin{document}
\maketitle

\section{Introduction}
In this project, I've tried to face the Max-Mean Dispersion Problem using Tabu
Search guided by a deep reinforcement learning algorithm during the dispersion phase.\\
Given a complete graph G(V, E) where each edge has associated a distance or
affinity $d_{ij}$, the Max Mean Dispersion Problem is the search of a subset of
vertex $M \subset V, |M| >= 2$ which maximize the mean dispersion, calculated as
follow:
$$
MeanDispersion(M) = \frac{\sum_{i<j; i,j \in M}d_{ij}}{|M|}
$$
This problem is known to be strongly NP-hard \cite{prokopyev2009equitable} and
has the characteristic that the subset $M$ does not have a fixed size, but can
vary between $2$ and $|V|$.
% Qua forse posso scrivere altro, parlare del modello MILP, che può essere usato
% sotto le 100 sol e ad altri metodi/heuristiche usate per attaccare il problema
% In in 2013, Martin and Sandoya proposed the Greedy Random Adaptive Search integrated with the Path
% Relinking Method (GRASP-PR) [18]. It uses a randomized greedy mechanism to
% maintain first building elite solutions (ES) and a variable neighborhood descent
% procedure for improvement.
\section{RLTS}
In 2020, Nijimbere et al. proposed an approach based on the
combination of reinforcement learning and tabu search, named RLTS. The main idea is to use
Q-Learning to build a high-quality initial solution, then improving it with a one-flip tabu search.\\
Then Q-Learning algorithm is trained during the search of the solution, by giving
a positive or negative reward if the node remain or not in the solution after
the tabu search.

\section{DQNTS}
My proposed solution is to use a Deep Q-Network to generate the initial solution
and tabu search to improve it, hence the name DQNTS.
The approach is similar to
\cite{nijimbere2020tabu}, the network learns a heuristic to build the initial
solution, then uses a one-flip tabu search to improve that solution.
The difference is that the DQN is pre-trained, hence there is no need to start
with a random solution and I can generate a high-quality initial solution from
the beginning.

\subsection{Network Architecture}
The network architecture is based on \cite{dai2017learning}, the implementation
and the hyperparameters settings can be seen on
\href{https://github.com/LucaLumetti/DQNTS}{github.com}. The state2tens embedding is
done with 4 features extracted from each node, which are:
\begin{itemize}
  \item{1 if the node in the solution, 0 otherwise}
  \item{the sum of all edges connected to the node}
  \item{the sum of all edges connected to the node and the solution nodes}
  \item{the sum of all edges connected to the node and the nodes not in the solution}
\end{itemize}

\subsection{Network training}
To train the network, first, we construct a feasible solution using an
$\epsilon$-greedy strategy, stopping when no positive rewards are predicted by
the network.  Then the solution is given to the one-flip tabu search which
improves it and returns the best solution. For every node in the initial solution which remains
in the final solution after tabu search, a reward of +1 is given, otherwise, the
reward is -1. The network has been trained over 10 different instances (MDPIx\_35
and MDPIIx\_35, $1 <= x <= 10$) for 5001 episodes.

\subsection{Tabu Search}
The tabu search implementation is the same as in \cite{nijimbere2020tabu}, with the only
difference in the parameter $\alpha = 100$ instead of $\alpha = 50000$. My
implementation couldn't finish even a single iteration in the time limit imposed
with the parameter $\alpha$ proposed in the paper. This made me think that my
implementation is way slower, but still, I left the same
time constraints as the results were good enough.\\

\subsection{General Algorithm}
DQN is used during the construction of the initial solution:
for all the nodes, the network estimates the reward, then all the values
estimated get interpolated in the range $[-1,+1]$. All the nodes with a value
$>= 0$ are named as "good nodes". Among these "good nodes", a random amount is
taken to construct the initial solution. Picking one node at a time would result
in a better solution, but this approach was too slow for large graphs. This
solution is then processed with a one-flip tabu search until no best solutions are
found for $\alpha = 100$ iterations in a row.  Now a new initial solution is
generated and the process is repeated until the time limit is not
violated. Finally, the best solution found is returned.

\section{Results}
The algorithm has been tested over the same 60 instances of
\cite{nijimbere2020tabu}, downloaded from
\href{http://grafo.etsii.urjc.es/optsicom/edp/}{here} and I've compared the
results against the ones obtained by RLTS. The time limit has been set to 10s
for $N <= 150$, 100s for $500 <= N <= 1000$, 1000s for $N = 3000$ and 2000s for
$N = 5000$.\\
The RLTS algorithm seems to perform better on smaller graphs while my algorithm
seems to perform better on larger graphs. The word "seems" is used here because
the random nature of my algorithm can lead to different results when executed
multiple times. To have more reliable results, more tests over the same
instances should be made. This has been done in \cite{nijimbere2020tabu}, but
that would require me a computational time that I don't have.\\
The CPU used by the RLTS algorithm is an Intel Xeon Processor E5-2670 @ 2.5GHz, 20 core, and the GPU is Nvidia Tesla K20m
I tested my algorithm on Intel Core i7-3610QM @ 2.30GHz, 4 core, and no GPU.
The better results on larger graphs probably come from a better first initial solution
than RLTS, which is fundamental on graphs of this size as the tabu search part
take nearly all the time limit.\\

\begin{center}
  \begin{tabular}{ l|l|l|l|l }
    \hline
    \textbf{Instance} & \textbf{N} & \textbf{MAMMDP} & \textbf{RLTS} & \textbf{DQNTS} \\ \hline
    MDPI1\_150 & 150 & 45.920 & 0 & 45.920 \\ \hline
    MDPI2\_150 & 150 & 43.392 & 0 & 53.615 \\ \hline
    MDPI3\_150 & 150 & 40.046 & 0 & 40.551 \\ \hline
    MDPI4\_150 & 150 & 44.044 & 0 & 44.255 \\ \hline
    MDPI5\_150 & 150 & 42.479 & 0 & 44.027 \\ \hline
    MDPI6\_150 & 150 & 43.723 & 0 & 48.868 \\ \hline
    MDPI7\_150 & 150 & 46.077 & 0 & 41.905 \\ \hline
    MDPI8\_150 & 150 & 42.451 & 0 & 47.971 \\ \hline
    MDPI9\_150 & 150 & 42.480 & 0 & 38.179 \\ \hline
    MDPI10\_150 & 150 & 41.798 & 0 & 41.798 \\ \hline
    \hline
    MDPI1\_500 & 500 & 81.277 & 81.277  & 63.982 \\ \hline
    MDPI2\_500 & 500 & 78.610 & 78.610  & 79.050 \\ \hline
    MDPI3\_500 & 500 & 76.301 & 76.301  & 70.006 \\ \hline
    MDPI4\_500 & 500 & 82.332 & 82.332  & 84.137 \\ \hline
    MDPI5\_500 & 500 & 80.354 & 80.354  & 81.011 \\ \hline
    MDPI6\_500 & 500 & 81.249 & 81.249  & 85.059 \\ \hline
    MDPI7\_500 & 500 & 78.165 & 78.165  & 72.869 \\ \hline
    MDPI8\_500 & 500 & 79.140 & 79.140  & 74.365 \\ \hline
    MDPI9\_500 & 500 & 77.421 & 77.421  & 68.911 \\ \hline
    MDPI10\_500 & 500 & 81.310 & 81.310  & 81.310 \\ \hline
    \hline
    MDPI1\_750 & 750 & 96.651 & 96.651 & 95.385 \\ \hline
    MDPI2\_750 & 750 & 97.565 & 97.565 & 98.667 \\ \hline
    MDPI3\_750 & 750 & 97.799 & 97.799 & 97.924 \\ \hline
    MDPI4\_750 & 750 & 96.041 & 96.041 & 94.651 \\ \hline
    MDPI5\_750 & 750 & 96.762 & 96.762 & 88.796 \\ \hline
    MDPI6\_750 & 750 & 99.861 & 99.861 & 103.78 \\ \hline
    MDPI7\_750 & 750 & 96.545 & 96.545 & 72.990 \\ \hline
    MDPI8\_750 & 750 & 96.727 & 96.727 & 95.034 \\ \hline
    MDPI9\_750 & 750 & 98.058 & 98.058 & 66.766 \\ \hline
    MDPI10\_750 & 750 & 100.060 & 100.060 & 100.005 \\ \hline
    \hline
    MDPI1\_1000 & 1000 & 119.174 & 119.174 & 117.817 \\ \hline
    MDPI2\_1000 & 1000 & 113.525 & 113.525 & 113.417 \\ \hline
    MDPI3\_1000 & 1000 & 115.139 & 115.139 & 101.471 \\ \hline
    MDPI4\_1000 & 1000 & 111.150 & 111.150 & 103.959 \\ \hline
    MDPI5\_1000 & 1000 & 112.723 & 112.723 & 84.118 \\ \hline
    MDPI6\_1000 & 1000 & 113.199 & 113.199 & 111.463 \\ \hline
    MDPI7\_1000 & 1000 & 111.556 & 111.556 & 109.773 \\ \hline
    MDPI8\_1000 & 1000 & 111.263 & 111.263 & 109.865 \\ \hline
    MDPI9\_1000 & 1000 & 115.959 & 115.959 & 112.635 \\ \hline
    MDPI10\_1000 & 1000 & 114.731 & 114.731 & 113.717 \\ \hline
    \hline

  \end{tabular}
\end{center}


\bibliography{main}
\bibliographystyle{plain}
\end{document}
