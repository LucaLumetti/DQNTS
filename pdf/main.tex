\documentclass{article}
\usepackage{amsmath}
\usepackage{enumerate}
\usepackage[framemethod=tikz]{mdframed}

\usepackage{listings}
\lstset{
	basicstyle=\ttfamily,
}

\usepackage{geometry}

\geometry{
	paper=a4paper, % Paper size, change to letterpaper for US letter size
	top=2.5cm, % Top margin
	bottom=3cm, % Bottom margin
	left=2.5cm, % Left margin
	right=2.5cm, % Right margin
	headheight=14pt, % Header height
	footskip=1.5cm, % Space from the bottom margin to the baseline of the footer
	headsep=1.2cm, % Space from the top margin to the baseline of the header
	% showframe, % Uncomment to show how the type block is set on the page
}

%----------------------------------------------------------------------------------------
%	FONTS
%----------------------------------------------------------------------------------------

\usepackage[utf8]{inputenc} % Required for inputting international characters
\usepackage[T1]{fontenc} % Output font encoding for international characters

% \usepackage{XCharter} % Use the XCharter fonts

%----------------------------------------------------------------------------------------
%	COMMAND LINE ENVIRONMENT
%----------------------------------------------------------------------------------------

% Usage:
% \begin{commandline}
%	\begin{verbatim}
%		$ ls
%		
%		Applications	Desktop	...
%	\end{verbatim}
% \end{commandline}

\mdfdefinestyle{commandline}{
	leftmargin=10pt,
	rightmargin=10pt,
	innerleftmargin=15pt,
	middlelinecolor=black!50!white,
	middlelinewidth=2pt,
	frametitlerule=false,
	backgroundcolor=black!5!white,
	frametitle={Command Line},
	frametitlefont={\normalfont\sffamily\color{white}\hspace{-1em}},
	frametitlebackgroundcolor=black!50!white,
	nobreak,
}

% Define a custom environment for command-line snapshots
\newenvironment{commandline}{
	\medskip
	\begin{mdframed}[style=commandline]
}{
	\end{mdframed}
	\medskip
}

%----------------------------------------------------------------------------------------
%	FILE CONTENTS ENVIRONMENT
%----------------------------------------------------------------------------------------

% Usage:
% \begin{file}[optional filename, defaults to "File"]
%	File contents, for example, with a listings environment
% \end{file}

\mdfdefinestyle{file}{
	innertopmargin=1.6\baselineskip,
	innerbottommargin=0.8\baselineskip,
	topline=false, bottomline=false,
	leftline=false, rightline=false,
	leftmargin=2cm,
	rightmargin=2cm,
	singleextra={%
		\draw[fill=black!10!white](P)++(0,-1.2em)rectangle(P-|O);
		\node[anchor=north west]
		at(P-|O){\ttfamily\mdfilename};
		%
		\def\l{3em}
		\draw(O-|P)++(-\l,0)--++(\l,\l)--(P)--(P-|O)--(O)--cycle;
		\draw(O-|P)++(-\l,0)--++(0,\l)--++(\l,0);
	},
	nobreak,
}

% Define a custom environment for file contents
\newenvironment{file}[1][File]{ % Set the default filename to "File"
	\medskip
	\newcommand{\mdfilename}{#1}
	\begin{mdframed}[style=file]
}{
	\end{mdframed}
	\medskip
}

%----------------------------------------------------------------------------------------
%	NUMBERED QUESTIONS ENVIRONMENT
%----------------------------------------------------------------------------------------

% Usage:
% \begin{question}[optional title]
%	Question contents
% \end{question}

\mdfdefinestyle{question}{
	innertopmargin=1.2\baselineskip,
	innerbottommargin=0.8\baselineskip,
	roundcorner=5pt,
	nobreak,
	singleextra={%
		\draw(P-|O)node[xshift=1em,anchor=west,fill=white,draw,rounded corners=5pt]{%
		Question \theQuestion\questionTitle};
	},
}

\newcounter{Question} % Stores the current question number that gets iterated with each new question

% Define a custom environment for numbered questions
\newenvironment{question}[1][\unskip]{
	\bigskip
	\stepcounter{Question}
	\newcommand{\questionTitle}{~#1}
	\begin{mdframed}[style=question]
}{
	\end{mdframed}
	\medskip
}

%----------------------------------------------------------------------------------------
%	WARNING TEXT ENVIRONMENT
%----------------------------------------------------------------------------------------

% Usage:
% \begin{warn}[optional title, defaults to "Warning:"]
%	Contents
% \end{warn}

\mdfdefinestyle{warning}{
	topline=false, bottomline=false,
	leftline=false, rightline=false,
	nobreak,
	singleextra={%
		\draw(P-|O)++(-0.5em,0)node(tmp1){};
		\draw(P-|O)++(0.5em,0)node(tmp2){};
		\fill[black,rotate around={45:(P-|O)}](tmp1)rectangle(tmp2);
		\node at(P-|O){\color{white}\scriptsize\bf !};
		\draw[very thick](P-|O)++(0,-1em)--(O);%--(O-|P);
	}
}

% Define a custom environment for warning text
\newenvironment{warn}[1][Warning:]{ % Set the default warning to "Warning:"
	\medskip
	\begin{mdframed}[style=warning]
		\noindent{\textbf{#1}}
}{
	\end{mdframed}
}

%----------------------------------------------------------------------------------------
%	INFORMATION ENVIRONMENT
%----------------------------------------------------------------------------------------

% Usage:
% \begin{info}[optional title, defaults to "Info:"]
% 	contents
% 	\end{info}

\mdfdefinestyle{info}{%
	topline=false, bottomline=false,
	leftline=false, rightline=false,
	nobreak,
	singleextra={%
		\fill[black](P-|O)circle[radius=0.4em];
		\node at(P-|O){\color{white}\scriptsize\bf i};
		\draw[very thick](P-|O)++(0,-0.8em)--(O);%--(O-|P);
	}
}

% Define a custom environment for information
\newenvironment{info}[1][Info:]{ % Set the default title to "Info:"
	\medskip
	\begin{mdframed}[style=info]
		\noindent{\textbf{#1}}
}{
	\end{mdframed}
}


\usepackage[ruled,vlined]{algorithm2e}

\title{Automated Decision Making: Final Project}
\author{Luca Lumetti\\ \texttt{244577@studenti.unimore.it}}

\begin{document}
\maketitle

\section{Introduction}
In this project I've tried to face the Max-Mean Dispersion Problem using Tabu
Search guided by deep reinforcement learning during the dispersion fase.\\
Given a complete graph G(V, E) where each edge has associated a distance or
affinity $d_{ij}$, the Max Mean Dispersion Problem is the search of a subset of
vertex $M \subset V, |M| >= 2$ which maximize the dispersion, calcuated as
follow:
$$
MeanDispersion(M) = \frac{\sum_{i<j; i,j \in M}d_{ij}}{|M|}
$$
% Qua forse posso scrivere altro, parlare del modello MILP, che può essere usato
% sotto le 100 sol e ad altri metodi/heuristiche usate per attaccare il problema
% In in 2013, Martin and Sandoya proposed the Greedy Random Adaptive Search integrated with the Path
% Relinking Method (GRASP-PR) [18]. It uses a randomized greedy mechanism to
% maintain first building elite solutions (ES) and a variable neighborhood descent
% procedure for improvement.
\section{RLTS}
In 2020, Nijimbere et al. proposed an approach based on the
combination of reinforcement learning and tabu search, named RLTS. The main idea is to use
Q-Learning to build an high-quality initial solution, then the
initial solution is improved with a one-flip tabu search algorithm.\\

\section{DQNTS}
The main idea is to let a network to learn an heuristich to build the initial
solution using Deep Q-Learning, which can generalize to graphs of any size, then use one-flip tabu
search to improve that solution as in \cite{nijimbere2020tabu}. In this case the
Q-Learning algorithm should generate better solutions than RLTS at the initial interations.

\subsection{Network Architecture}
The network architecture is based on \cite{nijimbere2020tabu}, the
hyperparameters setting can be seen at www.github.com. The state2tens embedding
is done with 4 features extracted from each node, which are:
\begin{itemize}
  \item{1 if the node in in the solution, 0 otherwise}
  \item{the sum of all edges connected to the node}
  \item{the sum of all edges connected to the node and the solution nodes}
  \item{the sum of all edges connected to the node and the nodes not in the solution}
\end{itemize}

\subsection{Network training}
To train the network, first we construct a feasible solution using an
$\epsilon$-greedy strategy, stopping when no positive rewards are predicted by
the network.  Then the solution is given to the one-flip tabu search which
return the best solution.  For every node in the initial solution which remains
in the final solution after tabu search, a reward of +1 is given, otherwise the
reward is -1 The network has been trained over 10 different instances (MDPIx\_35
and MDPIIx\_35, $1 <= x <= 10$) for 5001 episodes.

\subsection{Tabu Search}
The tabu search implementation is the same as in \cite{nijimbere2020tabu}, with the only
difference in the parameter $\alpha = 100$ instead of $\alpha = 50000$. My
implementation couldn't finish even a single iteration in the time limit imposed
with the parameter $\alpha$ proposed in the paper. This made me think that my
implementation is way slower, but still i left the same
time contraints as the results were good enough.\\

\subsection{General Algorithm}
The network architecture is used during the contruction of the inital solution:
for all the nodes, the network estimate the reward, then all the values get
interpolated in the range $[-1,+1]$ and all the nodes $>= 0$ are named as "good
nodes". Among these "good nodes", a random amount is taken to construct the
initial solution. Then this solution is processed with one-flip tabu search
until no best solutions are found for $\alpha = 100$ iterations in a row.
Now a new initial solution is generated and the process is repeated again, until
the time limit is not violated. Finally the best solution found is returned.

% \section{Results}
% \begin{center}
  \begin{tabular}{ c|c|c|c|c }
    \hline
    \textbf{Instance} & \textbf{N} & \textbf{RLTS} & \textbf{DQNTS} & \textbf{Obj. Gap} \\ \hline
    MDPI1\_150 & 150 & 45.920 & 45.920 & 0.000 \\ \hline
    MDPI2\_150 & 150 & \textbf{43.392} & 43.386 & 0.006 \\ \hline
    MDPI3\_150 & 150 & \textbf{40.046} & 40.037 & 0.011 \\ \hline
    MDPI4\_150 & 150 & 44.044 & 44.044 & 0.000 \\ \hline
    MDPI5\_150 & 150 & 42.479 & 42.479 & 0.000 \\ \hline
    MDPI6\_150 & 150 & 43.723 & 43.723 & 0.000 \\ \hline
    MDPI7\_150 & 150 & 46.077 & 46.077 & 0.000 \\ \hline
    MDPI8\_150 & 150 & 42.451 & 42.451 & 0.000 \\ \hline
    MDPI9\_150 & 150 & 42.480 & 42.480 & 0.000 \\ \hline
    MDPI10\_150 & 150 & 41.798 & 41.798 & 0.000 \\ \hline
  \end{tabular}
  \bigskip
  \begin{tabular}{ c|c|c|c|c }
    \hline
    \textbf{Instance} & \textbf{N} & \textbf{RLTS} & \textbf{DQNTS} & \textbf{Obj. Gap} \\ \hline
    MDPI1\_500 & 500 & \textbf{81.277} & 81.134 & 0.143 \\ \hline
    MDPI2\_500 & 500 & 78.610 & 78.610 & 0.000 \\ \hline
    MDPI3\_500 & 500 & \textbf{76.301} & 76.087 & 0.214 \\ \hline
    MDPI4\_500 & 500 & \textbf{82.332} & 82.229 & 0.103 \\ \hline
    MDPI5\_500 & 500 & \textbf{80.354} & 80.195 & 0.150 \\ \hline
    MDPI6\_500 & 500 & 81.249 & 81.249 & 0.000 \\ \hline
    MDPI7\_500 & 500 & \textbf{78.165} & 77.323 & 0.842 \\ \hline
    MDPI8\_500 & 500 & \textbf{79.140} & 78.931 & 0.209 \\ \hline
    MDPI9\_500 & 500 & \textbf{77.421} & 77.352 & 0.069 \\ \hline
    MDPI10\_500 & 500 & 81.310 & 81.183 & 0.000 \\ \hline
  \end{tabular}
  \bigskip
  \begin{tabular}{ c|c|c|c|c }
    \hline
    \textbf{Instance} & \textbf{N} & \textbf{RLTS} & \textbf{DQNTS} & \textbf{Obj. Gap} \\ \hline
    MDPI1\_750 & 750 & \textbf{96.651} & 94.949 & 1.702 \\ \hline
    MDPI2\_750 & 750 & \textbf{97.565} & 94.333 & 3.232 \\ \hline
    MDPI3\_750 & 750 & \textbf{97.799} & 96.980 & 0.819 \\ \hline
    MDPI4\_750 & 750 & \textbf{96.041} & 95.166 & 0.875 \\ \hline
    MDPI5\_750 & 750 & \textbf{96.762} & 95.455 & 1.307 \\ \hline
    MDPI6\_750 & 750 & \textbf{99.861} & 98.651 & 1.210 \\ \hline
    MDPI7\_750 & 750 & \textbf{96.545} & 95.178 & 1.397 \\ \hline
    MDPI8\_750 & 750 & \textbf{96.727} & 95.401 & 1.326 \\ \hline
    MDPI9\_750 & 750 & \textbf{98.058} & 97.394 & 0.664 \\ \hline
    MDPI10\_750 & 750 & \textbf{100.060} & 100.005 & 0.055 \\ \hline
  \end{tabular}
  \bigskip
  \begin{tabular}{ c|c|c|c|c }
    \hline
    \textbf{Instance} & \textbf{N} & \textbf{RLTS} & \textbf{DQNTS} & \textbf{Obj. Gap} \\ \hline
    MDPI1\_1000 & 1000 & \textbf{119.174} & 117.273 & 1.901 \\ \hline
    MDPI2\_1000 & 1000 & \textbf{113.525} & 113.417 & 0.108 \\ \hline
    MDPI3\_1000 & 1000 & \textbf{115.139} & 113.680 & 1.459 \\ \hline
    MDPI4\_1000 & 1000 & \textbf{111.150} & 109.882 & 1.268 \\ \hline
    MDPI5\_1000 & 1000 & \textbf{112.723} & 112.723 & 0.000 \\ \hline
    MDPI6\_1000 & 1000 & \textbf{113.199} & 110.309 & 2.810 \\ \hline
    MDPI7\_1000 & 1000 & \textbf{111.556} & 109.730 & 1.826 \\ \hline
    MDPI8\_1000 & 1000 & \textbf{111.263} & 110.369 & 0.894 \\ \hline
    MDPI9\_1000 & 1000 & \textbf{115.959} & 114.812 & 1.147 \\ \hline
    MDPI10\_1000 & 1000 & \textbf{114.731} & 113.717 & 1.014 \\ \hline
  \end{tabular}
  \bigskip
  \begin{tabular}{ c|c|c|c|c }
    \hline
    \textbf{Instance} & \textbf{N} & \textbf{RLTS} & \textbf{DQNTS} & \textbf{Obj. Gap} \\ \hline
    MDPI1\_5000 & 5000 & 240.159 & \textbf{245.624} & 5.465 \\ \hline
    MDPI2\_5000 & 5000 & 241.827 & \textbf{245.669} & 3.842 \\ \hline
    MDPI3\_5000 & 5000 & 240.890 & \textbf{245.858} & 4.968 \\ \hline
    MDPI4\_5000 & 5000 & 240.997 & \textbf{248.330} & 7.333 \\ \hline
    MDPI5\_5000 & 5000 & 242.480 & \textbf{245.765} & 3.315 \\ \hline
    MDPI6\_5000 & 5000 & 240.329 & \textbf{246.188} & 5.859 \\ \hline
    MDPI7\_5000 & 5000 & 242.820 & \textbf{245.399} & 2.579 \\ \hline
    MDPI8\_5000 & 5000 & 241.195 & \textbf{244.498} & 3.303 \\ \hline
    MDPI9\_5000 & 5000 & \textbf{239.761} & 211.580 & 28.181 \\ \hline
    MDPI10\_5000 & 5000 & 243.474 & \textbf{247.482} & 4.008 \\ \hline
  \end{tabular}
\end{center}


\bibliography{main}
\bibliographystyle{plain}
\end{document}
